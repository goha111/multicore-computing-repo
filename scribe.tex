%
% This is the LaTeX template file for lecture notes for EE 382C/EE 361C.
%
% To familiarize yourself with this template, the body contains
% some examples of its use.  Look them over.  Then you can
% run LaTeX on this file.  After you have LaTeXed this file then
% you can look over the result either by printing it out with
% dvips or using xdvi.
%
% This template is based on the template for Prof. Sinclair's CS 270.

\documentclass[twoside]{article}
\usepackage{graphicx}
\setlength{\oddsidemargin}{0.25 in}
\setlength{\evensidemargin}{-0.25 in}
\setlength{\topmargin}{-0.6 in}
\setlength{\textwidth}{6.5 in}
\setlength{\textheight}{8.5 in}
\setlength{\headsep}{0.75 in}
\setlength{\parindent}{0 in}
\setlength{\parskip}{0.1 in}

%
% The following commands set up the lecnum (lecture number)
% counter and make various numbering schemes work relative
% to the lecture number.
%
\newcounter{lecnum}
\renewcommand{\thepage}{\thelecnum-\arabic{page}}
\renewcommand{\thesection}{\thelecnum.\arabic{section}}
\renewcommand{\theequation}{\thelecnum.\arabic{equation}}
\renewcommand{\thefigure}{\thelecnum.\arabic{figure}}
\renewcommand{\thetable}{\thelecnum.\arabic{table}}

%
% The following macro is used to generate the header.
%
\newcommand{\lecture}[4]{
   \pagestyle{myheadings}
   \thispagestyle{plain}
   \newpage
   \setcounter{lecnum}{#1}
   \setcounter{page}{1}
   \noindent
   \begin{center}
   \framebox{
      \vbox{\vspace{2mm}
    \hbox to 6.28in { {\bf EE 382C/361C: Multicore Computing
                        \hfill Fall 2016} }
       \vspace{4mm}
       \hbox to 6.28in { {\Large \hfill Lecture #1: #2  \hfill} }
       \vspace{2mm}
       \hbox to 6.28in { {\it Lecturer: #3 \hfill Scribe: #4} }
      \vspace{2mm}}
   }
   \end{center}
   \markboth{Lecture #1: #2}{Lecture #1: #2}
   %{\bf Disclaimer}: {\it These notes have not been subjected to the
   %usual scrutiny reserved for formal publications.  They may be distributed
   %outside this class only with the permission of the Instructor.}
   \vspace*{4mm}
}

%
% Convention for citations is authors' initials followed by the year.
% For example, to cite a paper by Leighton and Maggs you would type
% \cite{LM89}, and to cite a paper by Strassen you would type \cite{S69}.
% (To avoid bibliography problems, for now we redefine the \cite command.)
% Also commands that create a suitable format for the reference list.
\renewcommand{\cite}[1]{[#1]}
\def\beginrefs{\begin{list}%
        {[\arabic{equation}]}{\usecounter{equation}
         \setlength{\leftmargin}{2.0truecm}\setlength{\labelsep}{0.4truecm}%
         \setlength{\labelwidth}{1.6truecm}}}
\def\endrefs{\end{list}}
\def\bibentry#1{\item[\hbox{[#1]}]}

%Use this command for a figure; it puts a figure in wherever you want it.
%usage: \fig{NUMBER}{SPACE-IN-INCHES}{CAPTION}
\newcommand{\fig}[3]{
			\vspace{#2}
			\begin{center}
			Figure \thelecnum.#1:~#3
			\end{center}
	}
% Use these for theorems, lemmas, proofs, etc.
\newtheorem{theorem}{Theorem}[lecnum]
\newtheorem{lemma}[theorem]{Lemma}
\newtheorem{proposition}[theorem]{Proposition}
\newtheorem{claim}[theorem]{Claim}
\newtheorem{corollary}[theorem]{Corollary}
\newtheorem{definition}[theorem]{Definition}
\newenvironment{proof}{{\bf Proof:}}{\hfill\rule{2mm}{2mm}}

% **** IF YOU WANT TO DEFINE ADDITIONAL MACROS FOR YOURSELF, PUT THEM HERE:

\begin{document}
%FILL IN THE RIGHT INFO.
%\lecture{**LECTURE-NUMBER**}{**DATE**}{**LECTURER**}{**SCRIBE**}
\lecture{24}{November 17th}{Vijay Garg}{Hang GONG}
%\footnotetext{These notes are partially based on those of Nigel Mansell.}

% **** YOUR NOTES GO HERE:

% Some general latex examples and examples making use of the
% macros follow.  
%**** IN GENERAL, BE BRIEF. LONG SCRIBE NOTES, NO MATTER HOW WELL WRITTEN,
%**** ARE NEVER READ BY ANYBODY.
\section{Introduction}
Some questions for Chapter 5.

\section*{Q1}
In SRSW multi-value register, how do we get the correct value?
\begin{enumerate}
\item[A.] do a single forward scan until we find a true bit
\item[B.] do a single backward scan until we find a true bit
\item[C.] first do a forward scan, then do a backward scan
\end{enumerate}

Answer: C

\section*{Q2}
why this code is considered safe instead of regular? (In which scenario it is not regular?)
\begin{center}
	\centering
	\includegraphics[width=4.5in]{img1.png}
\end{center}

Answer: If setValue and getValue are invoked concurrently, the value being written is the same as the previous value and getValue returns a different value.

\section*{Q3}
Which register we use a communication matrix for the construction?
\begin{enumerate}
\item[A.] SRSW  multi-value register
\item[B.] MRSW multi-value register
\item[C.] MRMW multi-value register
\end{enumerate}

Answer: B


\section*{Q4}
Which register we use a technique like Lamport's Bakery Algorithm for the construction?
\begin{enumerate}
\item[A.] SRSW  multi-value register
\item[B.] MRSW multi-value register
\item[C.] MRMW multi-value register
\end{enumerate}


Answer: C


\section*{Q5}
In a \textit{safe} register, if the initial value is 0, and one read operation overlaps with one write(3) (write 3 to the register) operation. Which value can be possibly returned by the read? (multiple choice)

\begin{enumerate}
\item[A.] 0
\item[B.] 3
\item[C.] 100
\end{enumerate}

Answer: ABC  (if read overlaps with write, a safe register can return \textbf{anything})


\section*{Q6}
In a \textit{regular} register, if the initial value is 0, and one read operation overlaps with one write(3) (write 3 to the register) operation. Which value can be possibly returned by the read? (multiple choice)

\begin{enumerate}
\item[A.] 0
\item[B.] 3
\item[C.] 100
\end{enumerate}

Answer: AB (if read overlaps with write, a \textit{regular} register can return either the value of the most recent write that preceded the read or the value of one of the overlapping writes.)

\section*{Q7}
When we do a write operation in SRSW multi-valued register, when we do a write operation, how do we update the array bits that preceding it?

\begin{enumerate}
\item[A.] In a backward direction, update the array entries from here to zero
\item[B.] In a forward direction, update the array from zero to here
\item[C.] Never mind, either will be OK
\end{enumerate}

Answer: A

\section*{References}
\beginrefs
\bibentry{1}{\sc V.K. Garg},  Introduction to Multicore Computing
\endrefs


\end{document}
